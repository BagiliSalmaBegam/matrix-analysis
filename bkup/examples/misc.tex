
\renewcommand{\theequation}{\theenumi}
%\begin{enumerate}[label=\arabic*.,ref=\theenumi]
\begin{enumerate}[label=\thesubsection.\arabic*.,ref=\thesubsection.\theenumi]
\numberwithin{equation}{enumi}
\item  Given unit basis vectors $\vec{a}, \vec{b}$, with angle	$\theta $ between them, the locus of the coordinates of a unit vector $\vec{c}$ in the space spanned by $\vec{a}, \vec{b}$  is given by 
		\begin{align}
\vec{x}^{\top}  \myvec{1 & \rho \\ \rho & 1}\vec{x} = 1
			\label{eq:misc-c-ellipse}
		\end{align}
			with $\rho = \cos \theta $.
\\
%		\begin{align}
%		\vec{V} = 	\myvec{1 & \cos \theta\\ \cos \theta & 1}, 
%			\label{eq:misc-locus}
%		\end{align}
		\solution Let 

		\begin{align}
			\vec{c}& =x_1\vec{a}+ x_2\vec{b} = \myvec{\vec{a} & \vec{b}}\vec{x}
		\end{align}
		Then, 
		\begin{align}
		\norm{\vec{c}}^2 &= \vec{x}^{\top}\myvec{\vec{a}^{\top} \\ \vec{b}^{\top}}\myvec{\vec{a} & \vec{b}}\vec{x}
\\
			&= \vec{x}^{\top}\myvec{1 & \vec{a}^{\top} \vec{b} \\ \vec{a}^{\top}\vec{b}& 1}\vec{x}
		\end{align}
			which can be expressed as 
			\eqref{eq:misc-c-ellipse}.
		\item Given the coordinates of $\vec{c}$, the angle $\theta$ between the  basis vectors 
			is given by
		\begin{align}
			\rho 
			= \frac{1 - \norm{\vec{x}}^2}{\vec{x}^{\top}\vec{R}\vec{x}}
			\label{eq:misc-c-ellipse-d-rho-final}
		\end{align}
		where 
		\begin{align}
			\vec{R}	&= \myvec{0 & 1 \\ 1 & 0}.
		\end{align}
%			by $\rho = \cos \theta$ in 
%			\eqref{eq:misc-c-ellipse-d-rho-final}.

		\solution Let 
		\begin{align}
			\vec{x} = \myvec{x_1 \\ x_2}
		\end{align}
For 
	  \begin{align}
		  \vec{V} = \myvec{1 & \rho \\ \rho & 1}, \vec{u} = 0, f = -1
	  \end{align}
	  in 
  \eqref{eq:conic_quad_form}, 
%		and 
		%\begin{align}
		%	\vec{x} = \myvec{x_1 \\ x_2}
		%\end{align}

		Since
		\begin{align}
			\mydet{\vec{V}} = 1-\rho^2, 
			0 < \mydet{\vec{V}} < 1,
		\end{align}
\eqref{eq:misc-c-ellipse}
		represents the equation of an ellipse.
		Using eigenvalue decomposition, 
		\begin{align}
			\vec{V} = \vec{P}^{\top}\vec{D}\vec{P}
		\end{align}
		where 
		\begin{align}
			\vec{P} = \frac{1}{\sqrt{2}}\myvec{1 & 1 \\ -1 & 1},
			\vec{D} = \myvec{1+\rho & 0 \\ 0 & 1-\rho}
		\end{align}
		Using the affine transformation, 
		\begin{align}
			\vec{x} = 
			\vec{P}\vec{y}
			\label{eq:misc-c-ellipse-affine}
		\end{align}
			\eqref{eq:misc-c-ellipse}
			can be expressed as 
		\begin{align}
			\vec{y}^{\top}\vec{D}\vec{y} &= 1
			\label{eq:misc-c-ellipse-d}
			\implies y_1^2\brak{1+\rho} + y_2^2\brak{1-\rho} &= 1
		\end{align}
		which can be simplified to obtain 
		\begin{align}
			\rho &= \frac{1 - y_1^2 -  y_2^2}{y_1^2 -  y_2^2}
			\\
			&= \frac{1 - \norm{\vec{y}}^2}{\vec{y}^{\top}\vec{Q}\vec{y}}
			\label{eq:misc-c-ellipse-d-rho}
		\end{align}
		where 
		\begin{align}
			\vec{Q}	&= \myvec{1 & 0 \\ 0 & -1}
		\end{align}
			From \eqref{eq:misc-c-ellipse-affine}, 
			\eqref{eq:misc-c-ellipse-d-rho} can be expressed as 
			\eqref{eq:misc-c-ellipse-d-rho-final}
			%\eqref{eq:misc-locus}.
		\begin{align}
\because \norm{\vec{a}} = \norm{\vec{b}} = 1.
		\end{align}

%		\begin{align}
%			\rho &= 			\cos \theta = \vec{a}^{\top} \vec{b} 
%\\
%			\implies \vec{x}^{\top}\vec{V}\vec{x} &= 1
%		\end{align}
%		where
%		\\
%			\label{eq:misc-c}
%			\\
%			&\brak
%		\end{align}
%		Letting  be the angle between $\vec{a}, \vec{b}$, 
%		\begin{align}
%			\rho &= 			\cos \theta = \vec{a}^{\top} \vec{b} 
%\\
%			\implies \vec{V} &= \myvec{1 & \rho \\ \rho & 1}, \abs{\rho} < 1
%		\end{align}
%		From 
%			\eqref{eq:misc-c}, since $\norm{\vec{c}} = 1$, 
