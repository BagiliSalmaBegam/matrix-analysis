
%We provide alternative solutions to the examples in \cite[p.336-352]{loney_coord}
%using the framework developed in this paper.
\begin{example}[parabola]
	To show that 
	%\cite[p.336]{loney_coord}
	\begin{align}
		9x^2 - 24xy + 16y^2-18x-101y+19=0
    \label{ex:parab}
	\end{align}
	is the equation of a parabola with latus rectum of length 3, vertex 
	\begin{align}
    \label{ex:parab-vertex}
		\frac{1}{25}\myvec{-29\\25}
	\end{align}
	and
	axis 
	\begin{align}
    \label{ex:parab-axis}
    3x-4y+7 = 0
	\end{align}
\end{example}
\solution 
	Comparing 
    \eqref{ex:parab}
with 
    \eqref{eq:conic_quad_form}, 
\begin{align}
    \label{ex:parab-V}
	\vec{V} &=\myvec{9 & -12 \\ -12 & 16}
\\
    \label{ex:parab-u}
	\vec{u} &= -\frac{1}{2}\myvec{18 \\ 101}
\\
    \label{ex:parab-f}
f &= 19
    \end{align}
    The eigenvalues of $\vec{V}$ are obtained as 
   % the solutions of 
\begin{align}
	\mydet{\lambda\vec{I}-\vec{V}} &=0
%	\mydet{\lambda-9 & 12 \\ 12 & \lambda-16} = 0
%	\\
	\\
	\implies 	
	%\lambda\brak{\lambda-25} &= 0 
	 \lambda_1 = 0, \lambda_2 = 25
	 \label{ex:parab-lam}
    \end{align}
    Since the $\vec{V}$ matrix has a 0 eigenvalue, 
    \eqref{ex:parab} is a parabola.
    The eigenvector corresponding to the 0 eigenalue is given by 
    \begin{align}
	    \myvec{9 & -12 \\ -12 & 16}\vec{p}_1 = \vec{0}
    \end{align}
    yielding
%    Row reducing the above matrix yields
    \begin{align}
%	     \xleftrightarrow[]{R_2 \leftarrow 3R_2 +4R_1 }
%			    \myvec{
%				    9 & -12
%			    \\
%			    0 & 0
%		    }
%		    \\
 \vec{p}_1 = \frac{1}{5}\myvec{4 \\ 3} 
	    \label{eq:parab-evec}
    \end{align}
    Substituting from 
    \eqref{ex:parab-V},
    \eqref{ex:parab-u},
    \eqref{ex:parab-f},
	    \eqref{eq:parab-evec}
    and 
	 \eqref{ex:parab-lam}
 in \eqref{eq:f0},
      \eqref{eq:eta} and 
			\eqref{eq:latus-ellipse}
			the latus rectum is obtained as
    \begin{align}
	    \eta &=\frac{\abs{2\vec{u}^{\top}\vec{p}_1}}{\lambda_2}
	   % \frac{\myvec{18 & 101}\myvec{4 \\ 3}}{25\times 5} 
	    = 3
    \end{align}
    The vertex of the parabola is obtained from 
    \eqref{eq:conic_parab_c} as 
  \begin{align}
%	    \myvec{ -\frac{1}{2}\myvec{18 & 101}+\frac{3}{10}\myvec{4 & 3} \\ \myvec{9 & -12 \\ -12 & 16}
%}\vec{c} 
%	  &= 
%	  \myvec{-19 \\ \frac{3}{10}\myvec{4 \\ 3}+\frac{1}{2}\myvec{18 \\ 101}}  
%\\
%	  \implies 
	  \myvec{	  -39 & -73  \\9 & -12  \\-12 & 16 }\vec{c} &= 
	  \myvec{  -19\\-21\\28}
	  %\text{or, } \vec{c} = 
	  %\myvec{ - \frac{29}{25}\\ \frac{22}{25}}
	  \label{ex:parab-c}
    \end{align}	
    yielding
    \eqref{ex:parab-vertex}.
    The second eigenvector of $\vec{V}$ is orthogonal to $\vec{p}_1$ and obtained as 
    \begin{align}
 \vec{p}_2 = \frac{1}{5}\myvec{3 \\ -4} 
    \end{align}
    Substituting from 
    \eqref{ex:parab-vertex}
and
	  \eqref{ex:parab-c}
    in 
	  \eqref{eq:major-minor-axis-quad}, the equation of the axis of symmetry for the parabola can be expressed as 
%    \begin{align}	
%	    \vec{p}_2\brak{\vec{x}-\vec{c}} &= 0
%	    \\
%	    \implies 
%	   \myvec{3 & -4}  \brak{\vec{x}-
%		\frac{1}{25}\myvec{-29\\25}
%	    } &= 0
%    \end{align}	
%    yielding 
    \eqref{ex:parab-axis}.



\begin{example}[ellipse]
	To show that the equation 
	%\cite[p.340]{loney_coord}
	\begin{align}
\label{ex:ellipse}
		14x^2 - 4xy + 11y^2-44x-58y+71=0
	\end{align}
	represents an ellipse with  centre
	\begin{align}
	    \label{ex:ellipse-center}
		\vec{c} = \myvec{2 \\ 3}
	\end{align}
	and lengths of semi-axes
	\begin{align}
		\label{eq:ellipse-semi-axis}
		\sqrt{6} \text{ and } 2
	\end{align}
\end{example}
\solution   The parameters for
\eqref{ex:ellipse}, 
are 
\begin{align}
	    \label{ex:ellipse-V}
	\vec{V} &=\myvec{14 & -2 \\ -2 & 11}
\\
	    \label{ex:ellipse-u}
	\vec{u} &= \myvec{22 \\ 29}
\\
f &= 71
	    \label{ex:ellipse-f}
    \end{align}
    Since 
    \begin{align}
	    \mydet{\vec{V}} = 150 > 0,
    \end{align}
\eqref{ex:ellipse} is an ellipse.  
Substituting from 
	    \eqref{ex:ellipse-V}
	    and 
	    \eqref{ex:ellipse-u}
	    in
    \eqref{eq:conic_nonparab_c},
the center of the ellipse is obtained as 
%    \begin{align}
%	    \vec{c} =	    \frac{1}{150}\myvec{11 & 2 \\ 2& 14}  
%	    \myvec{22 \\ 29}
%    \end{align}
%    yielding 
	    \eqref{ex:ellipse-center}.
Also, the eigenvalues of 
$\vec{V}$
are 
\begin{align}
	\lambda_1 = 10, 	\lambda_2 = 15
	    \label{ex:ellipse-lam}
    \end{align}
%    with respective eigenvectors
%\begin{align}
%	\vec{p}_1 &= \frac{1}{\sqrt{5}}\myvec{ 1 \\ 2}
%	\\
%	\vec{p}_2 &= \frac{1}{\sqrt{5}}\myvec{ -2 \\ 1}
%    \end{align}
%From  
%		\eqref{eq:ellipse-semi-axis},
%the axes of the ellipse are given by 
%\begin{align}
%	\myvec{ 1 & 2} \brak{\vec{x} - \myvec{2\\3}} &=0
%	\\
%	\myvec{ 2 &  -1} \brak{\vec{x} - \myvec{2\\3}} &=0
%    \end{align}
%
    Substituting
from 
	    \eqref{ex:ellipse-V},
	    \eqref{ex:ellipse-u}
	    and
	    \eqref{ex:ellipse-f}
	    in
	  \eqref{eq:major-minor-axis-quad},
    \begin{align}
	    f_0 
%	    &= \myvec{22 & 29}
%	    \frac{1}{150}\myvec{11 & 2 \\ 2& 14}
%	    \myvec{22 \\ 29}
%	    \\
	    &= 60
	    \label{ex:ellipse-f0}
    \end{align}
    Substituting from 
	    \eqref{ex:ellipse-lam}
	    and 
	    \eqref{ex:ellipse-f0}
	    in 
\eqref{eq:chord-len-major}
and
\eqref{eq:chord-len-minor}, 
    the lengths of the semi-axes are obtained as 
		\eqref{eq:ellipse-semi-axis}.

\begin{example}[hyperbola]
To show that the equation 
	%\cite[p.341]{loney_coord}
	\begin{align}
\label{ex:hyper} 
		x^2 - 3xy + y^2+10x-10y+21=0
	\end{align}
\end{example}
%
represents a hyperbola with centre
	\begin{align}
		\vec{c} = \myvec{-2 \\ 2}
	\end{align}
	and length of semi-axes
	\begin{align}
		\label{eq:hyper-semi-axis}
		\sqrt{2} \text{ and } \sqrt{\frac{2}{5}}
	\end{align}
\solution   The conic parameters for
\eqref{ex:hyper}, 
are 
\begin{align}
	    \label{ex:hyper-V}
	\vec{V} &=\frac{1}{2}\myvec{2 & - {3}\\- {3} & 2}
\\
	    \label{ex:hyper-u}
	\vec{u} &= \myvec{5\\-5}
\\
f &= 21
	    \label{ex:hyper-f}
    \end{align}
    Since 
    \begin{align}
	    \mydet{\vec{V}}  =  - \frac{5}{4} < 0,
    \end{align}
\eqref{ex:hyper} is a hyperbola.  
Substituting from 
	    \eqref{ex:hyper-V}
	    and 
	    \eqref{ex:hyper-V}
	    in
    \eqref{eq:conic_nonparab_c},
the center of the hyperbola is obtained as 
%    \begin{align}
%	    \vec{c} =-\frac{2}{5}\myvec{ {2} &  {3}\\ {3} &  {2}}  \myvec{5\\-5}
%    \end{align}
%    yielding 
	    \eqref{ex:ellipse-center}.
Also, the eigenvalues of 
$\vec{V}$
are 
\begin{align}
	\lambda_1 = - \frac{1}{2}, 	\lambda_2 =
 \frac{5}{2}
	    \label{ex:hyper-lam}
    \end{align}
%    with respective eigenvectors
%\begin{align}
%	\vec{p}_1 &= 
%    \frac{1}{\sqrt{2}}\myvec{1 \\ 1}
%	\\
%	\vec{p}_2 &= 
%     \frac{1}{\sqrt{2}}\myvec{- 1\\ 1}
%    \end{align}
%From  
%		\eqref{eq:hyper-semi-axis},
%the axes of the hyper are given by 
%\begin{align}
%	\myvec{ 1 & 2} \brak{\vec{x} - \myvec{2\\3}} &=0
%	\\
%	\myvec{ 2 &  -1} \brak{\vec{x} - \myvec{2\\3}} &=0
%    \end{align}

From 
	  \eqref{eq:major-minor-axis-quad},
    \begin{align}
	    f_0 
%	    \myvec{22 & 29}
%	    \frac{1}{150}\myvec{11 & 2 \\ 2& 14}
%	    \myvec{22 \\ 29}
%	    \\
	    &= -1
	    \label{ex:hyper-f0}
    \end{align}
    Substituting from 
	    \eqref{ex:hyper-lam}
	    and 
	    \eqref{ex:hyper-f0}
	    in 
\eqref{eq:chord-len-major}
and
\eqref{eq:chord-len-minor}, 
    the lengths of the semi-axes are then obtained as 
		\eqref{eq:hyper-semi-axis}.
\begin{example}[tangents]
To show that the tangents to the curve 
	%\cite[p.352]{loney_coord}
	\begin{align}
\label{ex:tangents} 
		x^2 + 4xy + 3y^2-5x-6y+3=0
	\end{align}
	parallel to the line 
	\begin{align}
\label{ex:tangents-line} 
x+4y+c = 0
	\end{align}
	are 
	\begin{align}
\label{ex:tangents-lines} 
		\begin{split}
		x+4y-5 &= 0
		\\
		x+4y-8 &= 0
		\end{split}
	\end{align}
\end{example}
\solution   The conic parameters for
\eqref{ex:tangents} 
are 
\begin{align}
	    \label{ex:tangents-V}
	\vec{V} &=\myvec{1 & 2\\2 & 3 }
\\
	    \label{ex:tangents-u}
	\vec{u} &= \myvec{- \frac{5}{2}\\-3}
\\
f &= 3
	    \label{ex:tangents-f}
    \end{align}
    Since 
    \begin{align}
	    \mydet{\vec{V}} = -1 < 0,
    \end{align}
\eqref{ex:tangents} is a hyperbola.  
    From 
\eqref{ex:tangents-line}, the normal vector to the tangent is 
    \begin{align}
	    \vec{n} = \myvec{1\\4} 
	    \label{ex:tangents-normal}
    \end{align}
    The equation of the tangent can be expressed as 
    \begin{align}
	    \vec{n}^{\top} \brak{  \vec{x}- \vec{q}_i }&= 0
    \end{align}
    where $\vec{q}_i$ are the points of contact.  Comparing the above with 
\eqref{ex:tangents-line},
    \begin{align}
	    c &= -\vec{n}^{\top} \vec{q}_i 
	    \label{ex:tangents-c}
    \end{align}
    which, upon substituting from 
\eqref{eq:conic_tangent_qk} can be expressed as 
    \begin{align}
	    c &= -\vec{n}^{\top} \cbrak{
		    \vec{V}^{-1}\brak{\kappa_i \vec{n}-\vec{u}}}
	    \\
	    &= -\vec{n}^{\top} \cbrak{
		    \vec{V}^{-1}\brak{
\pm \sqrt{
\frac{
f_0
}
{
\vec{n}^{\top}\vec{V}^{-1}\vec{n}
}
}\vec{n}-\vec{u}}}
\\
	    &=  
\pm \sqrt{
f_0
\vec{n}^{\top}\vec{V}^{-1}\vec{n}
}
+\vec{n}^{\top}\vec{V}^{-1}\vec{u}.
%\\
%	    &= - 
%		    \brak{
%\pm \sqrt{
%f_0
%\vec{n}^{\top}\vec{V}^{-1}\vec{n}
%}
%	    +\vec{n}^{\top}\vec{c}}
    \end{align}
%    upon substituting from 
%\eqref{eq:conic_tangent_qk}
%and 
%    \eqref{eq:conic_nonparab_c}
Substituting from 
	    \eqref{ex:tangents-V},
	    \eqref{ex:tangents-u}
	    and 
	    \eqref{ex:tangents-f}
	    in
      \eqref{eq:f0}, 
    \begin{align}
	    \vec{f}_0 % &=
	   % \myvec{ \frac{5}{2}\\3}
	   % \myvec{-3 & 2\\2 & -1}\myvec{ \frac{5}{2}\\3} - 3
	   % \\
	    &=- \frac{3}{4}
	    \label{ex:tangents-f0}
    \end{align}
    Substituing from 
	    \eqref{ex:tangents-V},
	    \eqref{ex:tangents-u},
	    \eqref{ex:tangents-normal}
	    and 
	    \eqref{ex:tangents-f0}
	    in 
	\eqref{eq:conic_normal_k}, 
    \begin{align}
	    c = -5 \text{ or  } c = -8 
	    \label{ex:tangents-ci}
    \end{align}
    yielding 
\eqref{ex:tangents-lines}.
