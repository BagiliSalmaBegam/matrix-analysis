\iffalse
\documentclass[12pt]{article}
\usepackage{graphicx}
%\documentclass[journal,12pt,twocolumn]{IEEEtran}
\usepackage[none]{hyphenat}
\usepackage{graphicx}
\usepackage{listings}
\usepackage[english]{babel}
\usepackage{graphicx}
\usepackage{caption}
\usepackage[parfill]{parskip}
\usepackage{hyperref}
\usepackage{booktabs}
%\usepackage{setspace}\doublespacing\pagestyle{plain}
\def\inputGnumericTable{}
\usepackage{color}                                            %%
    \usepackage{array}                                            %%
    \usepackage{longtable}                                        %%
    \usepackage{calc}                                             %%
    \usepackage{multirow}                                         %%
    \usepackage{hhline}                                           %%
    \usepackage{ifthen}
\usepackage{array}
\usepackage{amsmath}   % for having text in math mode
\usepackage{parallel,enumitem}
\usepackage{listings}
\lstset{
language=tex,
frame=single,
breaklines=true
}
 
%Following 2 lines were added to remove the blank page at the beginning
\usepackage{atbegshi}% http://ctan.org/pkg/atbegshi
\AtBeginDocument{\AtBeginShipoutNext{\AtBeginShipoutDiscard}}
%
%New macro definitions
\newcommand{\mydet}[1]{\ensuremath{\begin{vmatrix}#1\end{vmatrix}}}
\providecommand{\brak}[1]{\ensuremath{\left(#1\right)}}
\providecommand{\norm}[1]{\left\lVert#1\right\rVert}
\newcommand{\solution}{\noindent \textbf{Solution: }}
\newcommand{\myvec}[1]{\ensuremath{\begin{pmatrix}#1\end{pmatrix}}}
\let\vec\mathbf
\begin{document}
\begin{center}
\enlargethispage{-4cm}
\title{\textbf{Straight Lines}}
\date{\vspace{-5ex}} %Not to print date automatically
\maketitle
\end{center}
\setcounter{page}{1}
\section*{11$^{th}$ Maths - Chapter 10}
This is Problem-1 from Exercise 10.4
\begin{enumerate}

\solution
\fi
The parameters of the given line are
\begin{align}
\vec{n}^{\top}\vec{x}=c \label{eq:chapters/11/10/4/1/2}
\end{align}
This equation can be expressed in the form of 
\begin{align}
\vec{n} = \myvec{k-3\\-4+k^2}, c  = -k^2+7k-6
\end{align}
\iffalse
Then \eqref{eq:chapters/11/10/4/1/1} can be expressed as
\begin{align}
\myvec{k-3 & -4+k^2}\vec{x} &=-k^2+7k-6\label{eq:chapters/11/10/4/1/4}
\end{align}
\fi
\begin{enumerate}
%part-1
    \item 
	    In this case,
	    \iffalse
The normal vector of $x$-axis is given by
\begin{align}
\myvec{0\\1}
\end{align}
\fi
equating $\vec{n}$ to the normal vector of $x$-axis,
\begin{align}
\myvec{k-3\\-4+k^2} &=\alpha\myvec{0\\1}\label{eq:chapters/11/10/4/1/6}
\\
\implies
k &=3
\end{align}
Substituting the value of $k$ in \eqref{eq:chapters/11/10/4/1/1}, the desired equation is
\begin{align}
        \myvec{0 & 5}\vec{x} &=6
\end{align}

\item In this case, 
equating $\vec{n}$ to the normal vector of $y$-axis,
\begin{align}
\myvec{k-3\\-4+k^2} &=\beta\myvec{1\\0}\label{eq:chapters/11/10/4/1/11}
\\
	\implies k &=\pm2
\end{align}
Substituting the value of $k$ in \eqref{eq:chapters/11/10/4/1/1}, the desired equation is 
\begin{align}
        \myvec{-1 & 0}\vec{x} &=4, \quad  k &=2\\
        \myvec{-5 & 0}\vec{x} &=-24, \quad  k &=-2
\end{align}
\item 
	In this case, 
\begin{align}
	c = 0 \implies 
	-k^2+7k-6 &= 0\\
	\implies k =1 \text{ or } k&=6
\end{align}
Substituting the value of $k$ in \eqref{eq:chapters/11/10/4/1/1}, the desired equations are 
\begin{align}
        \myvec{-2 & -3}\vec{x} &=0, \quad  k &=1\\
       \myvec{3 & 32}\vec{x} &=0, \quad  k &=6
\end{align}
\end{enumerate}
