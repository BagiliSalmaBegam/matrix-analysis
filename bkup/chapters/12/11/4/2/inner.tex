\iffalse
\documentclass[12pt]{article}
\setlength{\textwidth}{5cm}
\paperheight 12in
\paperwidth 12in
\usepackage{graphicx}
\usepackage[margin=0.2in]{geometry}
\usepackage{amsmath}
\usepackage{array}
\usepackage{booktabs}
\usepackage{listings}
\providecommand{\norm}[1]{\left\lVert#1\right\rVert}
\providecommand{\abs}[1]{\left\vert#1\right\vert}
\setlength{\arraycolsep}{2.5pt} % default: 5pt
\medmuskip = 1mu % default: 4mu plus 2mu minus 4mu
\usepackage{enumerate}
\let\vec\mathbf
\newcommand{\myvec}[1]{\ensuremath{\begin{pmatrix}#1\end{pmatrix}}}
\newcommand{\mydet}[1]{\ensuremath{\begin{vmatrix}#1\end{vmatrix}}}
\providecommand{\brak}[1]{\ensuremath{\left(#1\right)}}
\lstset{
frame=single,
breaklines=true,
columns=fullflexible
}
\title{\textbf{Line Assignment}}
\author{Srikanth}
\date{January 2023}
\begin{document}
\maketitle
\paragraph{\textit{Problem Statement} -
\section*{Solution}
\fi
Let 
\begin{align}
 \vec{A}=\myvec{l_1\\m_1\\n_1},\,\vec{B}=\myvec{l_2\\m_2\\n_2}, \,
 \vec{C}&=\myvec{m_1n_2-m_2n_1\\n_1l_2-n_2l_1\\l_1m_2-l_2m_1}.
 \end{align}
 Given that 
 \begin{align} 
 \vec{A}^{\top}\vec{B}=\vec{0},\, 
  \vec{A}^{\top}\vec{A}=\vec{1},\,
   \vec{B}^{\top}\vec{B}=\vec{1} 
 \end{align}
 Let 
\begin{align}
\vec{P}=\myvec{\vec{A} & \vec{B} & \vec{C}} 
	=\myvec{
l_1&l_2&m_1n_2-m_2n_1\\
        m_1&m_2&n_1l_2-n_2l_1\\
        n_1&n_2&l_1m_2-l_2m_1
}
	\end{align}
Then 	
	\begin{align}
	\vec{P}^{\top} \vec{P}=
%		\myvec{
%l_1^2+m_1^2+n_1^2&l_1l_2+m_1m_2+n_1n_2&l_1(m_1n_2-m_2n_1)+m_1(n_1l_2-n_2l_1)+n_1(l_1m_2-l_2m_1)\\
%l_1l_2+m_1n_2+n_1n_2&l_2^2+m_2^2+n_2^2&l_2(m_1n_2-m_2n_1)+m_2(n_1l_2-n_2l_1)+n_2(l_1m_2-l_2m_1)\\
%l_1(m_1n_2-n_2m_1)+m_1(n_1l_2-n_2l_1)\\+n_1(l_1m_2-l_2m_1)&l_2(m_1n_2-m_2n_1)+m_2(n_1l_2-n_2l_1)+n_2(l_1m_2-l_2m_1)&(l_1m_2-l_2m_1)^2+(n_1l_2-n_2l_1)^2+(m_1n_2-m_2n_1)^2
%}
\vec{I}
	\end{align}
	Hence, the three vectors are mutually perpendicular.
