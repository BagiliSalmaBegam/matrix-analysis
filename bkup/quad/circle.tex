%\subsection{The Quadratic Form}
%\numberwithin{equation}{subsection}
%\begin{enumerate}
\begin{enumerate}[label=\thesection.\arabic*.,ref=\thesection.\theenumi]
	\item The equation of a circle is given by 
	\label{prop:circ-eq}
\begin{align}
	\norm{\vec{x}}^2 + 2 \vec{u}^{\top}\vec{x} + f = 0
	\label{eq:circ-eq}
\end{align}
\item For a circle with centre $\vec{c}$ and radius r,
\begin{align}
	\vec{u} = -\vec{c}, f = \norm{\vec{u}}^2 - r^2
	\label{eq:circ-cr}
\end{align}
\item Any point $\vec{x}$ on a circle can be expressed as
\begin{align}
\vec{x} = \vec{c} + r\myvec{\cos \theta \\ \sin \theta}.
	\label{eq:circ-polar}
\end{align}
\item The equation of the common chord of intersection of two  circles is given by 
\begin{align}
	2\brak{\vec{u}_1
	   -\vec{u}_2}^{\top}\vec{x} + f_1 - f_2 = 0
	\label{eq:circ-chord}
\end{align}
\item The line joining the centre of a circle to the mid point of any chord is perpendicular to the chord.
	\label{prop:circ-chord-perp}
	\begin{proof}
	Let $AB$ be any chord of a circle with centre $\vec{O}= \vec{0}$ and radius $r$.  Then, 
\begin{align}
	\norm{\vec{A}}^2 
	=\norm{\vec{B}}^2  &= r^2
	\\
	\implies 
	\norm{\vec{A}}^2 
	-\norm{\vec{B}}^2  &=\vec{0}
	\\
	\text{or, }\brak{\vec{A}-\vec{B}}^{\top}\brak{\vec{A}+\vec{B}} &= \vec{0}
\end{align}
which can be expressed as 
\begin{align}
	\brak{\vec{A}-\vec{B}}^{\top}\brak{\frac{\vec{A}+\vec{B}}{2}-\vec{O}} = \vec{0}
\end{align}
	\end{proof}
\item Let 
\begin{align}
\vec{A} =  \myvec{\cos \theta_1 \\ \sin \theta_1},
\vec{B} =  \myvec{\cos \theta_2 \\ \sin \theta_2},
\end{align}
 be points on  a unit circle with centre $\vec{O}$ at the origin.  Then
\begin{align}
	\label{eq:circ-ang-centre}
	\cos AOB = \vec{A}^{\top}\vec{B} 
\end{align}
\item Let 
\begin{align}
\vec{A} =  \myvec{\cos \theta_1 \\ \sin \theta_1},
\vec{B} =  \myvec{\cos \theta_2 \\ \sin \theta_2},
\vec{C} =  \myvec{\cos \theta \\ \sin \theta},
\end{align}
 be points on  a unit circle.  Then
  \begin{align}
	\label{eq:circ-angle-1}
	  \cos ACB&= \frac{\brak{\vec{C}-\vec{A}}^{\top} \brak{\vec{C}-\vec{B}}}{\norm{\vec{C}-\vec{A}}\norm{\vec{C}-\vec{B}}}
	  \\
	  &= \cos \brak{\frac{\theta_1 - \theta_2}{2}}
	\label{eq:circ-angle-2}
  \end{align}
		\begin{proof}
			Since
\begin{align}
	\brak{\vec{C}-\vec{A}}^{\top} \brak{\vec{C}-\vec{B}} &= 
	\norm{\vec{C}}^2 - \vec{C}^{\top}\brak{\vec{A}+\vec{B}} + \vec{A}^{\top}\vec{B}
	\\
	&= 1 - \cos\brak{\theta-\theta_1} - \cos\brak{\theta-\theta_2} + \cos\brak{\theta_1-\theta_2}
	\\
	&= 2 \cos^2 \brak{\frac{\theta_1 - \theta_2}{2}}
	- 2 \cos \brak{\frac{\theta_1 - \theta_2}{2}}
	\cos \brak{\theta -\frac{\theta_1 + \theta_2}{2}}
	\\
	&= 4 \cos \brak{\frac{\theta_1 - \theta_2}{2}}
	\sin \brak{\frac{\theta - \theta_1}{2}}
	\sin \brak{\frac{\theta - \theta_2}{2}},
\end{align}
and 
\begin{align}
	\norm{\vec{C}-\vec{A}}^2 &= \norm{\vec{C}}^2+\norm{\vec{A}}^2 - 2\vec{C}^{\top}\vec{A},
	\\
&= 4 
	\sin^2 \brak{\frac{\theta - \theta_1}{2}}, 
	\\
	\norm{\vec{C}-\vec{B}}^2 &= \norm{\vec{C}}^2+\norm{\vec{B}}^2 - 2\vec{C}^{\top}\vec{B},
	\\
&= 4 
	\sin^2 \brak{\frac{\theta - \theta_2}{2}}, 
\end{align}
	\eqref{eq:circ-angle-1}
	can be expressed as
\begin{align}
	\frac{	  \cos \brak{\frac{\theta_1 - \theta_2}{2}}
	\sin \brak{\frac{\theta - \theta_1}{2}}
	\sin \brak{\frac{\theta - \theta_2}{2}}
	}
	{
	\sin \brak{\frac{\theta - \theta_1}{2}}
	\sin \brak{\frac{\theta - \theta_1}{2}} 
} 
\end{align}
yielding 
	\eqref{eq:circ-angle-2}
		\end{proof}
	\item From 
	\eqref{eq:circ-ang-centre}
	and 
	\eqref{eq:circ-angle-2},
	\label{prop:circ-ang-centre-arc}
\begin{align}
	\label{eq:circ-ang-centre-arc}
	\angle AOB = 2\angle AOC
\end{align}

\end{enumerate}
