\begin{enumerate}[label=\thesection.\arabic*,ref=\thesection.\theenumi]

	\item $\triangle\vec{A}\vec{B}\vec{C}$ with vertices $\vec{A}(-2,0), \vec{B}(2,0) \text{ and }\vec{C}(0,2)$ is similar to $\triangle \vec{DEF}$  with vertices $\vec {D}(-4,0),\vec{E}(4,0)  \text{ and } \vec{F}(0,4)$  
	\item Point $ (-4,2)$ lies on the line segment joining the points $ \vec{A}(-4,6) \text{ and } \vec{B}(-4,-6)$
 \item The points $(0,5),(0,-9)\text{ and }(3,6)$ are collinear
\item  Point $\vec{P}(0,2)$ is the point of intersection of $y$-axis and perpendicular bisector of line segment joining the points $\vec{A}(-1,1) \text{ and } \vec{B}(3,3)$
\item Points $\vec{A}(3,1), \vec{B}(12,-2) \text{ and } \vec {C}(0,2)$ cannot be the vertices of a triangle
\item Points $\vec{A}(4,3), \vec{B}(6,4),{c}(5,-6) \text{ and } \vec{D}(-3,5)$ are the vertices of a parallelogram  
\item A circie has its centre at the origin and a point $\vec{P}(5,0)$ lies on it The point $\vec{Q}(6,8)$ lies outside the circle
\item The point $\vec{A}(2,7)$ lies on the perpendicular bisector of line segment joining the points $\vec{P}(6,5)\text{ and } \vec{Q}(0,-4)$
\item Point $\vec{P}(5,-3)$ is one of the two points of trisection of line segment joining the points $\vec{A}(7,-2)\text{ and }\vec{B}(1,-5)$
\item Points $\vec{A}(-6,10),\vec{B}(-4,6) \text{ and } \vec{C}(3,-8)$ are collinear such that $\vec{A}\vec{B}=  \frac{2}{9}\vec{A}\vec{C}$
 \item The point $\vec{P}(-2,4)$lies on circie of radius 6 and center $\vec{C}(3,5)$
\item The points $\vec{A}(-1,-2),\vec{B}(4,3),\vec{C}(2,5) \text{ and } \vec{D}(-3,0)$ in that order a rectangle
	\end{enumerate}

