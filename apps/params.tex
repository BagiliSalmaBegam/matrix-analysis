
		\label{app:major}
		Since the major axis passes through the origin, 
  \begin{align}
	  \vec{q} =			\vec{0} 
\end{align}  
Further, from Corollary  
		\eqref{corr:axis},
  \begin{align}
  \vec{m}&= \vec{e}_2,  
\end{align} and
from 
    \eqref{eq:conic_simp_temp_nonparab},
  \begin{align}
	  \vec{V} =     \frac{\vec{D} }{f_0}, 
	   \vec{u} = 0, 
	   f = -1
	    \label{eq:latus_rectum_ellipse_param}
\end{align}  
Substituting the above in
\eqref{eq:chord-len}, 
\begin{align}
 \frac{2\sqrt{
\vec{e}_1^{\top}\frac{\vec{D}}{f_0}\vec{e}_1
}
}
{
\vec{e}_1^{\top}\frac{\vec{D}}{f_0}\vec{e}_1
}\norm{\vec{e}_1}
  \end{align}
  yielding 
\eqref{eq:chord-len-major}.
Similarly, for the minor axis, the only different parameter is 
  \begin{align}
  \vec{m}&= \vec{e}_2,  
\end{align} 
Substituting the above in
\eqref{eq:chord-len}, 
\begin{align}
 \frac{2\sqrt{
\vec{e}_2^{\top}\frac{\vec{D}}{f_0}\vec{e}_2
}
}
{
\vec{e}_2^{\top}\frac{\vec{D}}{f_0}\vec{e}_2
}\norm{\vec{e}_2}
  \end{align}
  yielding 
\eqref{eq:chord-len-minor}.
		\section{}
		\label{app:latus}
			The latus rectum is perpendicular to the major axis for the standard conic.  Hence, from Corollary  
		\eqref{corr:axis},
  \begin{align}
  \vec{m}&= \vec{e}_2,  
\end{align}  
Since it passes through the focus, from 
					\eqref{eq:F-ell-hyp-parab}
  \begin{align}
	  \vec{q} =			\vec{F} 
=
					 \pm e\sqrt{\frac{f_0}{\lambda_2\brak{1-e^2}}} \vec{e }_1
%					 \frac{e}{\sqrt{f_0\lambda_2\brak{1-e^2}}}\vec{e }_1
\end{align}  
for the standard hyperbola/ellipse.  Also, 
from 
    \eqref{eq:conic_simp_temp_nonparab},
  \begin{align}
	  \vec{V} =     \frac{\vec{D} }{f_0}, 
	   \vec{u} = 0, 
	   f = -1
	    \label{eq:latus_rectum_ellipse_param-new}
\end{align}  
Substituting the above in
\eqref{eq:chord-len}, 
\begin{align}
 \frac{2\sqrt{
\sbrak{
\vec{e}_2^{\top}\brak{\frac{\vec{D}}{f_0} e\sqrt{\frac{f_0}{\lambda_2\brak{1-e^2}}} \vec{e }_1}
}^2
-
\brak
{
 e\sqrt{\frac{f_0}{\lambda_2\brak{1-e^2}}} \vec{e }_1^{\top}\frac{\vec{D}}{f_0} e\sqrt{\frac{f_0}{\lambda_2\brak{1-e^2}}} \vec{e }_1 -1 
}
\brak{\vec{e}_2^{\top}\frac{\vec{D}}{f_0}\vec{e}_2}
}
}
{
\vec{e}_2^{\top}\frac{\vec{D}}{f_0}\vec{e}_2
}\norm{\vec{e}_2}
\label{eq:chord-len-sub-ell}
  \end{align}
  Since 
  \begin{align}
\vec{e}_2^{\top}\vec{D}\vec{e}_1 = 0, 
%\vec{e}_2^{\top}\vec{e}_2 = 0,
\vec{e}_1^{\top}\vec{D}\vec{e}_1 = \lambda_1,
\vec{e}_1^{\top}\vec{e}_1 = 1,
	  \norm{\vec{e}_2} = 1,
\vec{e}_2^{\top}\vec{D}\vec{e}_2 = \lambda_2,
  \end{align}
\eqref{eq:chord-len-sub-ell} can be expressed as 
  \begin{align}
	&		\frac{2\sqrt{\brak{1-\frac{\lambda_1e^2}{{\lambda_2\brak{1-e^2}}}}\brak{\frac{\lambda_2}{f_0}}}}
{
	\frac{\lambda_2}{f_0}
	} 	
	\\
	&=		2\frac{\sqrt{
		f_0\lambda_1}}{\lambda_2}
 & \brak{ \because e^2 = 1-\frac{\lambda_1}{\lambda_2}}
		   \end{align}
For the standard parabola, the parameters in 
\eqref{eq:chord-len} are
\begin{align}  
	\vec{q} =\vec{F} =  -\frac{\eta}{4\lambda_2}\vec{e}_1, \vec{m} = \vec{e}_1, \vec{V} = \vec{D},
	\vec{u} = \frac{\eta}{2}\vec{e}_1^{\top}, f = 0
\end{align}  

Substituting the above in
\eqref{eq:chord-len}, 
%			from \eqref{eq:conic_simp_temp_nonparab},  
%					from \eqref{eq:F-ell-hyp-parab}
%and 						 \\
the length of the latus rectum  can be expressed as
{\footnotesize
\begin{align}
 \frac{2\sqrt{
\sbrak{
\vec{e}_2^{\top}\brak{\vec{D}\brak{-\frac{\eta}{4\lambda_2}\vec{e}_1}+\frac{\eta}{2}\vec{e}_1}
}^2
-
\brak
{
\brak{-\frac{\eta}{4\lambda_2}\vec{e}_1}^{\top}\vec{D}\brak{-\frac{\eta}{4\lambda_2}\vec{e}_1} + 2\frac{\eta}{2}\vec{e}_1^{\top}\brak{-\frac{\eta}{4\lambda_2}\vec{e}_1} 
}
\brak{\vec{e}_2^{\top}\vec{D}\vec{e}_2}
}
}
{
\vec{e}_2^{\top}\vec{D}\vec{e}_2
}\norm{\vec{e}_2}
\label{eq:chord-len-sub}
  \end{align}
  }
  Since 
  \begin{align}
\vec{e}_2^{\top}\vec{D}\vec{e}_1 = 0, 
\vec{e}_2^{\top}\vec{e}_2 = 0,
\vec{e}_1^{\top}\vec{D}\vec{e}_1 = 0,
\vec{e}_1^{\top}\vec{e}_1 = 1,
	  \norm{\vec{e}_1} = 1,
\vec{e}_2^{\top}\vec{D}\vec{e}_2 = \lambda_2,
  \end{align}
\eqref{eq:chord-len-sub} can be expressed as 
  \begin{align}
	  2 \frac{\sqrt{\frac{\eta^2}{4\lambda_2}\lambda_2}}{\lambda_2}
	  = \frac{\eta}{\lambda_2}
  \end{align}

%
%\end{enumerate}
