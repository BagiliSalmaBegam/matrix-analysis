
\begin{theorem}
Using the affine transformation in
\eqref{eq:conic_affine},
	the conic in     \eqref{eq:conic_quad_form} can be expressed in standard form 
	%(centre/vertex at the origin, major axis - $x$ axis)
	as
  \begin{align}
    %\begin{aligned}
    \label{eq:conic_simp_temp_nonparab}
	    \vec{y}^{\top}\brak{\frac{\vec{D}}{f_0}}\vec{y} &= 1   &  \abs{\vec{V}} &\ne 0
    \\
	    \vec{y}^{\top}\vec{D}\vec{y} &=  -\eta\vec{e}_1^{\top}\vec{y}   & \abs{\vec{V}} &= 0
    \label{eq:conic_simp_temp_parab}
    %\end{aligned}
    \end{align}
    where
  \begin{align}
      %\begin{split}
      \label{eq:f0}
	  f_0 &=\vec{u}^{\top}\vec{V}^{-1}\vec{u} -f \ne 0
	  \\
      \label{eq:eta}
       \eta &=2\vec{u}^{\top}\vec{p}_1
       \\
       \vec{e}_1 &=\myvec{1 \\ 0}
      \end{align}
     
    \end{theorem}
\begin{proof}
  See Appendix \ref{app:parab}.
  \end{proof}
	  \begin{corollary}
		For the standard conic, 
				\begin{align}
					\label{eq:std-prm-P}
					\vec{P} &= \vec{I}
					\\
					\vec{u} &= 
				\begin{cases}
				0 & e \ne 1
       \\
				\frac{\eta}{2} \vec{e}_1 & e = 1
				\end{cases}
				\label{eq:std-prm-u}
				\\
				\lambda_1 &  
					\begin{cases}
						=0 & e = 1
						\\
						\ne 0 & e \ne 1
					\end{cases}
				\label{eq:std-prm-lam1}
				\end{align}
				where 
				\begin{align}
					\vec{I} = \myvec{\vec{e}_1 & \vec{e}_2}
				\end{align}
				is the identity matrix.
	  \end{corollary}
    \begin{theorem}\leavevmode
		\begin{enumerate}
			\item The directrices for the  standard conic are given by 
				\begin{align}
					\label{eq:dx-ell-hyp}
					\vec{e}_1^{\top}\vec{y} &=  
					%\pm\sqrt{\abs{\frac{f_0\lambda_2}{\lambda_1\brak{\lambda_2-\lambda_1}}}} & e \ne 1
					\pm \frac{1}{e}\sqrt{\frac{\abs{f_0}}{\lambda_2\brak{1-e^2}}} & e \ne 1
					\\
					\vec{e}_1^{\top}\vec{y} &= \frac{\eta}{2\lambda_2} & e = 1
					\label{eq:dx-parab}
				\end{align}
    \item The foci of the standard ellipse and hyperbola are given by 
				\begin{align}
					\label{eq:F-ell-hyp-parab}
					\vec{F} 
=
					\begin{cases}
						\pm e\sqrt{\frac{\abs{f_0}}{\lambda_2\brak{1-e^2}}}\vec{e}_1 & e \ne 1
					%	\pm \sqrt{\abs{\frac{f_0}{\lambda_1}\brak{1 - \frac{\lambda_1}{\lambda_2}}}}\vec{e}_1 & e \ne 1
						\\
						 -\frac{\eta}{4\lambda_2}\vec{e}_1 & e = 1
					\end{cases}
				\end{align}
	
		\end{enumerate}
	%	where, without loss of generality, $f_0 < 0$ for the hyperbola.
    \end{theorem}
	\begin{proof}%\leavevmode
		See Appendix \ref{app:foc-dir}.
	\end{proof}
