\begin{enumerate}
	\item
		The center/vertex of a conic section are given by
  \begin{align}
    \label{eq:conic_nonparab_c}
	    \vec{c} &= - \vec{V}^{-1}\vec{u}  & \mydet{\vec{V}} \ne 0
    \\
	    \myvec{ \vec{u}^{\top}+\frac{\eta}{2}\vec{p}_1^{\top} \\ \vec{v}}\vec{c} &= \myvec{-f \\ \frac{\eta}{2}\vec{p}_1-\vec{u}}  
& \mydet{\vec{V}} = 0
    \label{eq:conic_parab_c}
    \end{align}	
	
		\begin{proof}
			In 
			\eqref{eq:conic_affine}, substituting $\vec{y} = \vec{0}$, the center/vertex for the quadratic form is obtained as
    \begin{align}
	    \vec{x} = \vec{c}, 
    \end{align}
			where $\vec{c}$ is derived as 
    \eqref{eq:conic_nonparab_c}
    and 
    \eqref{eq:conic_parab_c}
in Appendix  \ref{app:parab}.
		\end{proof}

%
    \item The equation of the minor and major  axes for the ellipse/hyperbola are respectively given by 
  \begin{align}
\vec{p}_i^{\top}\brak{\vec{x}-\vec{c}} = 0, i = 1,2
	  \label{eq:major-minor-axis-quad}
  \end{align}
  The axis of symmetry for the parabola is also given by 
	  \eqref{eq:major-minor-axis-quad}.

		\begin{proof}
From		\eqref{corr:axis}, the major/symmetry axis for the hyperbola/ellipse/parabola can be expressed using 
\eqref{eq:conic_affine} as
  \begin{align}
	  \vec{e}_2^{\top}
		  \vec{P}^{\top}\brak{\vec{x}-\vec{c}} &= 0
		  \\
	  \implies 		  \brak{\vec{P}\vec{e}_2}^{\top}\brak{\vec{x}-\vec{c}} &= 0
  \end{align}
yielding	  \eqref{eq:major-minor-axis-quad}, and the proof for the minor axis is similar.
		\end{proof}
\end{enumerate}
