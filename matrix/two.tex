
\renewcommand{\theequation}{\theenumi}
%\begin{enumerate}[label=\arabic*.,ref=\theenumi]
\begin{enumerate}[label=\thesection.\arabic*.,ref=\thesection.\theenumi]
\numberwithin{equation}{enumi}
\item Let 
\begin{align}
  \vec{A} \equiv \overrightarrow{A} &= \myvec{a_1\\a_2} 
  \\
  &\equiv a_1\overrightarrow{i}+a_2\overrightarrow{j}, 
  \\
  \vec{B} &= \myvec{b_1\\b_2}, 
\end{align}
be $2 \times 1$ vectors.
Then, the determinant of the $2 \times 2$ matrix 
\begin{align}  
  \vec{M} = \myvec{\vec{A} & \vec{B}}
\end{align}
is defined as
\begin{align}
  \label{eq:det2d}
  \mydet{\vec{M}} &= \mydet{\vec{A} & \vec{B}} 
  \\
  &= \mydet{a_1 & b_1\\a_2 & b_2} = a_1b_2 - a_2 b_1
\end{align}
%
\item The value of the cross product of two vectors is given by  
  \eqref{eq:det2d}.
\item The area of the triangle with vertices $\vec{A}, \vec{B}, \vec{C}$ is given by the absolute value of 
\begin{align}
  \label{eq:area2d}
\frac{1}{2} \mydet{\vec{A-B} & \vec{A-C}}
  \end{align}
  \item  The transpose of $\vec{A}$ is defined as
\begin{align}
  \label{eq:transpose2d}
  \vec{A}^{\top}  = \myvec{a_1 & a_2}
\end{align}
%
\item The {\em inner product} or {\em dot product} is defined as
\begin{align}
  \label{eq:dot2d}
  \vec{A}^{\top} \vec{B} &\equiv \vec{A} \cdot \vec{B} 
  \\
  &= \myvec{a_1 & a_2} \myvec{b_1 \\ b_2}= a_1b_1+a_2b_2 
\end{align}
%
\item {\em norm} of $\vec{A}$ is defined as
\begin{align}
  \label{eq:norm2d}
  \norm{A} &\equiv \mydet{\overrightarrow{A}}
  \\
  &= \sqrt{\vec{A}^{\top} \vec{A}}= \sqrt{a_1^2+a_2^2}
\end{align}
Thus, 
\begin{align}
  \label{eq:norm2d_const}
  \norm{\lambda \vec{A}} &\equiv \mydet{\lambda\overrightarrow{A}}
  \\
  &= \abs{\lambda} \norm{\vec{A}}
\end{align}
\item The distance betwen the points $\vec{A}$ and $\vec{B}$ is given by 
\begin{align}
  \label{eq:norm2d_dist}
\norm{\vec{A}-\vec{B}} 
\end{align}
\item Let $\vec{x}$ be equidistant from the points $\vec{A}$ and $\vec{B}$.  Then 
  \begin{align}
	  \brak{\vec{A}-\vec{B}}^{\top}{\vec{x}} 
	  =  \frac{\norm{\vec{A}}^2 - \norm{\vec{B}}^2}{2}
  \label{eq:norm2d_equidist}
  \end{align}
  \solution 
\begin{align}
	\norm{\vec{x}-\vec{A}} &=
\norm{\vec{A}-\vec{B}} 
\\
	\implies \norm{\vec{x}-\vec{A}}^2 &=
\norm{\vec{x}-\vec{B}}^2 
\end{align}
which can be expressed as 
\begin{multline}
%  \label{eq:norm2d_dist}
	\brak{\vec{x}-\vec{A}}^{\top} \brak{\vec{x}-\vec{A}}=
	\brak{\vec{x}-\vec{B}}^{\top} 
\brak{\vec{x}-\vec{B}}
\\
	\implies	\norm{\vec{x}}^2-2{\vec{x}}^{\top}\vec{A} + \norm{\vec{A}}^2
	\\= \norm{\vec{x}}^2-2{\vec{x}}^{\top}\vec{B} + \norm{\vec{B}}^2
\end{multline}
which can be simplified to obtain
  \eqref{eq:norm2d_equidist}.
\item If $\vec{x}$ lies on the  $x$-axis and is  equidistant from the points $\vec{A}$ and $\vec{B}$, 
  \begin{align}
	  \vec{x} &=
	   x\vec{e}_1
  \end{align}
  where 
  \begin{align}
	  x &=\frac{\norm{\vec{A}}^2 -\norm{\vec{B}}^2 }{2\brak{\vec{A}-\vec{B}}^{\top }\vec{e}_1
}
	  \label{eq:cbse_10_x}
  \end{align}
  \solution 
  From \eqref{eq:norm2d_equidist}.
  \begin{align}
	   x\brak{\vec{A}-\vec{B}}^{\top }\vec{e}_1
		  &=
	  \frac{\norm{\vec{A}}^2 -\norm{\vec{B}}^2 }{2}
   \end{align}
	  yielding \eqref{eq:cbse_10_x}.
  \item The angle between two vectors is given by 
  \begin{align}
    \label{eq:angle2d}
    \theta = \cos^{-1}\frac{\vec{A}^{\top} \vec{B}}{\norm{A}\norm{B}}
  \end{align}
  \item If two vectors are orthogonal (perpendicular), 
  \begin{align}
    \label{eq:angle2d_orth}
\vec{A}^{\top} \vec{B} = 0
  \end{align}

  \item The {\em direction vector} of the line joining two points $\vec{A},\vec{B}$ is given by 
  \begin{align}
    \label{eq:dir_vec}
    \vec{m} = \vec{A}-\vec{B}
  \end{align}
\item The unit vector in the direction of $\vec{m}$ is defined as
\begin{align}
    \frac{\vec{m}}{\norm{\vec{m}}}
\end{align}
\item If the direction vector of a line is expressed as 
	\begin{align}
    \vec{m} = \myvec{1\\m},
\end{align}
 the $m$ is defined to be the {\em} slope of the line. 
  \item The {\em normal vector} to $\vec{m}$ is defined by 
  \begin{align}
    \label{eq:normal_vec}
    \vec{m}^{\top}  \vec{n} = 0
  \end{align}
  \item The point $\vec{P}$ that divides the line segment $AB$ in the ratio $k:1$  is given by 

  \begin{align}
	  \vec{P}&= \frac{k\vec{B}+ \vec{A}}{k+1}
	  \label{eq:section_formula}
  \end{align}
\item  The standard basis vectors are defined as 

  \begin{align}
  \vec{e}_1&= \myvec{1\\0}, 
  \\
  \vec{e}_2&= \myvec{0\\1}.
  \end{align}
\end{enumerate}
