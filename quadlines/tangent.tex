
\begin{theorem}[Tangent]
  Given the point of contact $\vec{q}$, the equation of a tangent to \eqref{eq:conic_quad_form} is 
  \begin{align}
  \brak{\vec{V}\vec{q}+\vec{u}}^{\top}\vec{x}+\vec{u}^{\top}\vec{q}+f = 0
  \label{eq:conic_tangent_final}
  \end{align}
\end{theorem}
\begin{proof}
  The normal vector is obtained from \eqref{eq:conic_tangent_mq} and \eqref{eq:normal_vec}
  as
  %
  \begin{align}
  \label{eq:conic_normal_vec}
	  \kappa \vec{n} = \vec{V}\vec{q}+\vec{u}, \kappa \in \mathbb{R}
  \end{align}  
  From \eqref{eq:conic_normal_vec} and \eqref{eq:line_norm_eq}, the equation of the tangent is\begin{align}
    \brak{\vec{V}\vec{q}+\vec{u}}^{\top}\brak{\vec{x}-\vec{q}} &=0
    \\
    \implies \brak{\vec{V}\vec{q}+\vec{u}}^{\top}\vec{x}-\vec{q}^{\top}\vec{V}\vec{q}-\vec{u}^{\top}\vec{q} &= 0
    \end{align}
    which, upon substituting from \eqref{eq:conic_tangent_qquad} and simplifying yields 
  \eqref{eq:conic_tangent_final}
%	\eqref{eq:conic_tangent}.
\end{proof}
\begin{theorem}
	\label{eq:conic-p-contact-nonparab}
  If $\vec{V}^{-1}$ exists, given the normal vector $\vec{n}$, the tangent points of contact to \eqref{eq:conic_quad_form} are given by
\begin{align}
  \begin{split}
\vec{q}_i &= \vec{V}^{-1}\brak{\kappa_i \vec{n}-\vec{u}}, i = 1,2
\\
\text{where }\kappa_i &= \pm \sqrt{
\frac{
f_0
%\vec{u}^{\top}\vec{V}^{-1}\vec{u}-f
}
{
\vec{n}^{\top}\vec{V}^{-1}\vec{n}
}
}
  \end{split}
\label{eq:conic_tangent_qk}
\end{align}
\end{theorem}
\begin{proof}
  From \eqref{eq:conic_normal_vec},
\begin{align}
\label{eq:conic_normal_vec_q}
 \vec{q} = \vec{V}^{-1}\brak{\kappa \vec{n}-\vec{u}}, \quad \kappa \in \mathbb{R}
\end{align}
Substituting \eqref{eq:conic_normal_vec_q}
in \eqref{eq:conic_tangent_qquad},
\begin{align}
\brak{\kappa \vec{n}-\vec{u}}^{\top}\vec{V}^{-1}\brak{\kappa \vec{n}-\vec{u}} 
%\\
+ 2\vec{u}^{\top}\vec{V}^{-1}\brak{\kappa \vec{n}-\vec{u}} +f &= 0
\\
\implies 
\kappa^2 \vec{n}^{\top}\vec{V}^{-1}\vec{n} - \vec{u}^{\top}\vec{V}^{-1}\vec{u} + f &=0
 \\
 \text{or, } \kappa = \pm \sqrt{\frac{
	 %\vec{u}^{\top}\vec{V}^{-1}\vec{u}-f
	f_0 
 }{\vec{n}^{\top}\vec{V}^{-1}\vec{n}}} &
	\label{eq:conic_normal_k}
\end{align}
%
%yileding 
Substituting \eqref{eq:conic_normal_k} in \eqref{eq:conic_normal_vec_q}
yields \eqref{eq:conic_tangent_qk}.
%
\end{proof}


\begin{theorem}
	\label{eq:conic-p-contact-parab}
  If $\vec{V}$ is not invertible,  given the normal vector $\vec{n}$, the point of contact to \eqref{eq:conic_quad_form} is given by the matrix equation
\begin{align}
\label{eq:conic_tangent_q_eigen}
\begin{pmatrix}
\vec{\brak{u+\kappa \vec{n}}}^{\top} \\ \vec{V}
\end{pmatrix}
\vec{q} &= 
\begin{pmatrix}
-f
\\
\kappa\vec{n}-\vec{u}
\end{pmatrix}
\\
\text{where }  \kappa = \frac{\vec{p}_1^{\top}\vec{u}}{\vec{p}_1^{\top}\vec{n}}, \quad \vec{V}\vec{p}_1 &= 0
\label{eq:conic_tangent_qk_eigen}
\end{align}

\end{theorem}
\begin{proof}
  If $\vec{V}$ is non-invertible, it has a zero eigenvalue.  If the corresponding eigenvector is $\vec{p}_1$, then,
\begin{align}
\vec{V}\vec{p}_1 = 0
\label{eq:conic_zero_eigen}
\end{align}
From \eqref{eq:conic_normal_vec},
\begin{align}
\label{eq:conic_zero_eigen_normal}
\kappa \vec{n} &= \vec{V} \vec{q}+\vec{u}, \quad \kappa \in \mathbb{R}
\\
\implies \kappa \vec{p}_1^{\top}\vec{n} &= \vec{p}_1^{\top}\vec{V} \vec{q}+\vec{p}_1^{\top}\vec{u}
\\
\text{or, } \kappa \vec{p}_1^{\top}\vec{n} &= \vec{p}_1^{\top}\vec{u},  \quad \because \vec{p}_1^{\top} \vec{V} = 0, 
%\\
\quad 
\brak{\text{ from } \eqref{eq:conic_zero_eigen}}
%\label{eq:conic_normal_vec_q}
\end{align}
yielding $\kappa$ in \eqref{eq:conic_tangent_qk_eigen}. From \eqref{eq:conic_zero_eigen_normal},
\begin{align}
\kappa \vec{q}^{\top}\vec{n} &= \vec{q}^{\top}\vec{V} \vec{q}+\vec{q}^{\top}\vec{u}
\\
\implies \kappa \vec{q}^{\top}\vec{n} &= -f-\vec{q}^{\top}\vec{u} \quad \text{from } \eqref{eq:conic_tangent_qquad},
\\
\text{or, } \brak{\kappa \vec{n}+\vec{u}}^{\top}\vec{q} &= -f
\label{eq:conic_zero_eigen_normal_fq}
\end{align}
\eqref{eq:conic_zero_eigen_normal} can be expressed as
\begin{align}
\label{eq:conic_zero_eigen_normal_vq}
\vec{V} \vec{q} = \kappa \vec{n} - \vec{u}.
\end{align}
\eqref{eq:conic_zero_eigen_normal_fq} and \eqref{eq:conic_zero_eigen_normal_vq} clubbed together result in \eqref{eq:conic_tangent_q_eigen}.
\end{proof}
\begin{theorem}
	The normal vectors of the tangents from 
	a point $\vec{h}$ to the conic in \eqref{eq:conic_quad_form}
	are given by 
\begin{align}
	\label{eq:dual-nh}
	\vec{n} = \frac{\vec{e}_1}{\vec{e}_1^{\top}\vec{h}}+\mu_i\vec{R}\vec{h}, 
\end{align}
where
\begin{align}
	\vec{R} &= 
	\vec{R}_{\frac{\pi}{2}} = \myvec{0 & -1 \\ 1 & 0}
\end{align}
and $\mu_i$ are obtained from 
\eqref{eq:tangent_roots} by substituting
\begin{align}
	\vec{m} &= \vec{R}\vec{h}, \vec{u} = 0, \vec{q} = \frac{\vec{e}_1}{\vec{e}_1^{\top}\vec{h}}
\end{align}
	and replacing $\vec{V}$ with $\vec{V}^{-1}$.

\end{theorem}
%
\begin{proof}
From 
  \eqref{eq:conic_tangent_mq}, the normal vector to  the tangent at $\vec{q}$ can be expressed as 
  \begin{align}
  \vec{n} &= \vec{V}\vec{q}+\vec{u} 
  \label{eq:conic_normal_n}
  \\
  \implies \vec{q} &= \vec{V}^{-1}\brak{\vec{n}-\vec{u} }
  \label{eq:conic_normal_q}
  \end{align}
  which upon substituting in \eqref{eq:conic_quad_form} yields
\begin{align}
    \label{eq:conic_quad_form_q}
    \brak{\vec{n}-\vec{u} }^{\top}\vec{V}^{-1}\vec{V}\vec{V}^{-1}\brak{\vec{n}-\vec{u} }+2\vec{u}^{\top}\vec{V}^{-1}\brak{\vec{n}-\vec{u} }+f&=0
    \\
    \implies \vec{n} ^{\top}\vec{V}^{-1}\vec{n}-f_0 = 0
	\label{eq:dual-nf0}
	%\vec{u}^{\top}\vec{V}^{-1}\vec{u} +f&=0
    \end{align}
	Since the tangents pass through the point $\vec{h}$, 
    \begin{align}
\vec{n}^{\top}\vec{h} = 1
    \end{align}
and	from 
	\eqref{eq:mat-nh}, we obtain 
	\eqref{eq:dual-nh}.
\end{proof}
\begin{theorem}
	The normal vectors of the tangents from 
	a point $\vec{h}$ to the conic in \eqref{eq:conic_quad_form}
	are given by 
  \begin{align} 
  \label{eq:quad_form_pair_normvecs-sigma}
  \begin{split}
  \vec{n}_1 &= \vec{P}\myvec{\sqrt{\abs{\lambda_1}} \\[2mm]  \sqrt{\abs{\lambda_2}}}
  \\
  \vec{n}_2 &= \vec{P}\myvec{\sqrt{\abs{\lambda_1}} \\[2mm] - \sqrt{\abs{\lambda_2}}}
  \end{split}
  \end{align} 
  where $\lambda_i, \vec{P}$ are the eigenparameters of 
  \begin{align} 
		\bm{\Sigma} &= 
	   \brak{\vec{V}\vec{h}+\vec{u}}
	  \brak{\vec{V}\vec{h}+\vec{u}}^{\top}
   -\vec{V}
  \brak
  {
  \vec{h}^{\top}\vec{V}\vec{h} + 2\vec{u}^{\top}\vec{h} +f
  }.
	  \label{eq:h-tangents-sigma}
  \end{align}                    
\end{theorem}
\begin{proof}
 From \eqref{eq:tangent_roots},
 and
  \eqref{eq:conic_tangent_disc}
  \begin{align}
  \sbrak{
  \vec{m}^{\top}\brak{\vec{V}\vec{h}+\vec{u}}
  }^2 -\brak{\vec{m}^{\top}\vec{V}\vec{m}}
  \brak
  {
  \vec{h}^{\top}\vec{V}\vec{h} + 2\vec{u}^{\top}\vec{h} +f
  } &= 0                                                                                             
  \\
	  \implies 
  \vec{m}^{\top}  \sbrak{\brak{\vec{V}\vec{h}+\vec{u}}
	  \brak{\vec{V}\vec{h}+\vec{u}}^{\top}
   -\vec{V}
  \brak
  {
  \vec{h}^{\top}\vec{V}\vec{h} + 2\vec{u}^{\top}\vec{h} +f
  }}\vec{m} &= 0                                                                                             
  \label{eq:conic_tangent_disc-h}
  \end{align}                    
  yielding
	  \eqref{eq:h-tangents-sigma}.  Consequently, from 
  \eqref{eq:quad_form_pair_normvecs}, 
  \eqref{eq:quad_form_pair_normvecs-sigma}
  can be obtained.
\end{proof}
%
