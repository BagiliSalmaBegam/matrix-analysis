\iffalse
\documentclass[12pt]{article}
\usepackage{graphicx}
\usepackage{amsmath}
\usepackage{mathtools}
\usepackage{gensymb}

\newcommand{\mydet}[1]{\ensuremath{\begin{vmatrix}#1\end{vmatrix}}}
\providecommand{\brak}[1]{\ensuremath{\left(#1\right)}}
\providecommand{\norm}[1]{\left\lVert#1\right\rVert}
\newcommand{\solution}{\noindent \textbf{Solution: }}
\newcommand{\myvec}[1]{\ensuremath{\begin{pmatrix}#1\end{pmatrix}}}
\let\vec\mathbf

\begin{document}
\begin{center}
\textbf\large{CHAPTER-10 \\ VECTOR ALGEBRA}

\end{center}
\section*{Excercise 10.3}

Q10.If $\vec{a} = 2\hat{i}+2\hat{j}+3\hat{k}, \vec{b} = -\hat{i}+2\hat{j}+\hat{k} \text{ and } \vec{c} = 3\hat{i}+\hat{j}$ are such that $\vec{a}+\lambda \vec{b}$ is perpendicular to $\vec{c}$, then find the value of $\lambda$.
\fi
\solution
Given that
\begin{align}
	(\vec{a}+\lambda \vec{b})^{\top} \vec{c} &= 0\\
\implies \vec{a}^{\top}\vec{c}+\lambda \vec{b}^{\top}\vec{c}&=0\\
\implies 	\lambda \vec{b}^{\top}\vec{c}&=-\vec{a}^{\top}\vec{c}\\
\implies 	\lambda(\vec{b}^{\top}\vec{c})(\vec{b}^{\top}\vec{c})^{-1}&=-(\vec{a}^{\top}\vec{c})(\vec{b}^{\top}\vec{c})^{-1}\\
\implies 	\lambda&=-(\vec{a}^{\top}\vec{c})(\vec{b}^{\top}\vec{c})^{-1}
\end{align}
Now substituting the values
\begin{align}
	\vec{a}^{\top}\vec{c}&=\myvec{2&2&3} \myvec{3\\1\\0} = 8\\
	\vec{b}^{\top}\vec{c}&=\myvec{-1&2&1} \myvec{3\\1\\0} = -1,
\end{align}
\begin{align}
	\lambda&=-(\vec{a}^{\top}\vec{c})(\vec{b}^{\top}\vec{c})^{-1}\\
	&=-(8)(-1)^{-1}\\
	&=8
\end{align}


