\documentclass[journal,12pt,twocolumn]{IEEEtran}
\usepackage{romannum}
\usepackage{float}
\usepackage{setspace}
\usepackage{gensymb}
\singlespacing
\usepackage[cmex10]{amsmath}
\usepackage{amsthm}
\usepackage{mathrsfs}
\usepackage{txfonts}
\usepackage{stfloats}
\usepackage{bm}
\usepackage{cite}
\usepackage{cases}
\usepackage{subfig}
\usepackage{longtable}
\usepackage{multirow}
\usepackage{enumitem}
\usepackage{mathtools}
\usepackage{steinmetz}
\usepackage{tikz}
\usepackage{circuitikz}
\usepackage{verbatim}
\usepackage{tfrupee}
\usepackage[breaklinks=true]{hyperref}
\usepackage{tkz-euclide}
\usetikzlibrary{calc,math}
\usepackage{listings}
    \usepackage{color}                                            %%
    \usepackage{array}                                            %%
    \usepackage{longtable}                                        %%
    \usepackage{calc}                                             %%
    \usepackage{multirow}                                         %%
    \usepackage{hhline}                                           %%
    \usepackage{ifthen}                                           %%
  %optionally (for landscape tables embedded in another document): %%
    \usepackage{lscape}     
\usepackage{multicol}
\usepackage{chngcntr}
\DeclareMathOperator*{\Res}{Res}
\renewcommand\thesection{\arabic{section}}
\renewcommand\thesubsection{\thesection.\arabic{subsection}}
\renewcommand\thesubsubsection{\thesubsection.\arabic{subsubsection}}

\renewcommand\thesectiondis{\arabic{section}}
\renewcommand\thesubsectiondis{\thesectiondis.\arabic{subsection}}
\renewcommand\thesubsubsectiondis{\thesubsectiondis.\arabic{subsubsection}}

% correct bad hyphenation here
\hyphenation{op-tical net-works semi-conduc-tor}
\def\inputGnumericTable{}                                 %%

\lstset{
frame=single, 
breaklines=true,
columns=fullflexible
}

\begin{document}


\newtheorem{theorem}{Theorem}[section]
\newtheorem{problem}{Problem}
\newtheorem{proposition}{Proposition}[section]
\newtheorem{lemma}{Lemma}[section]
\newtheorem{corollary}[theorem]{Corollary}
\newtheorem{example}{Example}[section]
\newtheorem{definition}[problem]{Definition}
\newcommand{\BEQA}{\begin{eqnarray}}
\newcommand{\EEQA}{\end{eqnarray}}
\newcommand{\define}{\stackrel{\triangle}{=}}

\bibliographystyle{IEEEtran}
\providecommand{\mbf}{\mathbf}
\providecommand{\pr}[1]{\ensuremath{\Pr\left(#1\right)}}
\providecommand{\qfunc}[1]{\ensuremath{Q\left(#1\right)}}
\providecommand{\sbrak}[1]{\ensuremath{{}\left[#1\right]}}
\providecommand{\lsbrak}[1]{\ensuremath{{}\left[#1\right.}}
\providecommand{\rsbrak}[1]{\ensuremath{{}\left.#1\right]}}
\providecommand{\brak}[1]{\ensuremath{\left(#1\right)}}
\providecommand{\lbrak}[1]{\ensuremath{\left(#1\right.}}
\providecommand{\rbrak}[1]{\ensuremath{\left.#1\right)}}
\providecommand{\cbrak}[1]{\ensuremath{\left\{#1\right\}}}
\providecommand{\lcbrak}[1]{\ensuremath{\left\{#1\right.}}
\providecommand{\rcbrak}[1]{\ensuremath{\left.#1\right\}}}
\theoremstyle{remark}
\newtheorem{rem}{Remark}
\newcommand{\sgn}{\mathop{\mathrm{sgn}}}
\providecommand{\abs}[1]{\left\vert#1\right\vert}
\providecommand{\res}[1]{\Res\displaylimits_{#1}} 
\providecommand{\norm}[1]{\left\lVert#1\right\rVert}
\providecommand{\mtx}[1]{\mathbf{#1}}
\providecommand{\mean}[1]{E\left[ #1 \right]}
\providecommand{\fourier}{\overset{\mathcal{F}}{ \rightleftharpoons}}
\providecommand{\system}{\overset{\mathcal{H}}{ \longleftrightarrow}}
\newcommand{\solution}{\noindent \textbf{Solution: }}
\newcommand{\cosec}{\,\text{cosec}\,}
\providecommand{\dec}[2]{\ensuremath{\overset{#1}{\underset{#2}{\gtrless}}}}
\newcommand{\myvec}[1]{\ensuremath{\begin{pmatrix}#1\end{pmatrix}}}
\newcommand{\mydet}[1]{\ensuremath{\begin{vmatrix}#1\end{vmatrix}}}
\numberwithin{equation}{subsection}
\makeatletter
\@addtoreset{figure}{problem}
\makeatother

\let\StandardTheFigure\thefigure
\let\vec\mathbf
\renewcommand{\thefigure}{\theproblem}



\def\putbox#1#2#3{\makebox[0in][l]{\makebox[#1][l]{}\raisebox{\baselineskip}[0in][0in]{\raisebox{#2}[0in][0in]{#3}}}}
     \def\rightbox#1{\makebox[0in][r]{#1}}
     \def\centbox#1{\makebox[0in]{#1}}
     \def\topbox#1{\raisebox{-\baselineskip}[0in][0in]{#1}}
     \def\midbox#1{\raisebox{-0.5\baselineskip}[0in][0in]{#1}}

\vspace{3cm}


\title{Assignment 1}
\author{Jaswanth Chowdary Madala}





% make the title area
\maketitle

\newpage

%\tableofcontents

\bigskip

\renewcommand{\thefigure}{\theenumi}
\renewcommand{\thetable}{\theenumi}

\begin{enumerate}
\item In the following cases, determine whether the given planes are parallel or perpendicular, and in case they are neither, find the angles between them.
\begin{enumerate}
\item $7x + 5y + 6z + 30 = 0$ and $3x – y – 10z + 4 = 0$
\item $2x + y + 3z – 2 = 0$ and $x – 2y + 5 = 0$
\item $2x – 2y + 4z + 5 = 0$ and $3x – 3y + 6z – 1 = 0$
\item $2x – y + 3z – 1 = 0$ and $2x – y + 3z + 3 = 0$
\item $4x + 8y + z – 8 = 0$ and $y + z – 4 = 0$
\end{enumerate}

\textbf{Solution:} The angle between the planes is the angle between the normals of the given planes.
\begin{align}
\vec{n_1}^{\top}\vec{x} = c_1, \, \vec{n_2}^{\top}\vec{x} &= c_2
\end{align}
The angle $\theta$ between the planes is given by,
\begin{align}
\cos{\theta} &= \frac{\vec{n_1}^{\top}\vec{n_2}}{\norm{\vec{n_1}}\norm{\vec{n_2}}}
\end{align}

\begin{enumerate}
\item
\begin{align}
\vec{n_1} = \myvec{7\\5\\6}, \, \vec{n_2} &= \myvec{3\\-1\\-10}\\
\vec{n_1}^{\top}\vec{n_2} &= \myvec{7&5&6}\myvec{3\\-1\\-10}\\
&= -44\\
\norm{\vec{n_1}} &= \sqrt{7^2+5^2+6^2}\\ 
&= \sqrt{110}\\
\norm{\vec{n_2}} &= \sqrt{3^2+\brak{-1}^2+\brak{-10}^2} \\
&= \sqrt{110}\\
\cos{\theta} &= -\frac{44}{\sqrt{110}\sqrt{110}}\\
&= -\frac{2}{5}
\end{align}
The planes are inclined at an angle of $\arccos\brak{{-\frac{2}{5}}}$ degrees.

\item
\begin{align}
\vec{n_1} = \myvec{2\\1\\3},\, \vec{n_2} &= \myvec{1\\-2\\0}\\
\vec{n_1}^{\top}\vec{n_2} &= \myvec{2&1&3}\myvec{1\\-2\\0}\\
&= 0\\
\cos{\theta} &= 0
\end{align}  
The planes are perpendicular.
\item
\begin{align}
\vec{n_1} = \myvec{2\\-2\\4},\, \vec{n_2} &= \myvec{3\\-3\\6}\\
\vec{n_1}^{\top}\vec{n_2} &= \myvec{2&-2&4}\myvec{3\\-3\\6}\\
&= 36\\
\norm{\vec{n_1}} &= \sqrt{2^2+\brak{-2}^2+4^2}\\ 
&= \sqrt{24}\\
\norm{\vec{n_2}} &= \sqrt{3^2+\brak{-3}^2+6^2} \\
&= \sqrt{54}\\
\cos{\theta}&= \frac{36}{\sqrt{24}\sqrt{54}}\\
&= 1
\end{align}
The planes are parallel.
\item
\begin{align}
\vec{n_1} = \myvec{2\\-1\\3}, \, \vec{n_2} &= \myvec{2\\-1\\3}\\
\vec{n_1}^{\top}\vec{n_2} &= \myvec{2&-1&3}\myvec{2\\-1\\3}\\
&= 14\\
\norm{\vec{n_1}} &= \sqrt{2^2+\brak{-1}^2+3^2}\\ 
&= \sqrt{14}\\
\norm{\vec{n_2}} &= \sqrt{2^2+\brak{-1}^2+3^2}\\
&= \sqrt{14}\\
\cos{\theta} &= \frac{14}{\sqrt{14}\sqrt{14}}\\
&= 1
\end{align}
The planes are parallel.
\item
\begin{align}
\vec{n_1} = \myvec{4\\8\\1}, \, \vec{n_2} &= \myvec{0\\1\\1}\\
\vec{n_1}^{\top}\vec{n_2} &= \myvec{4&8&1}\myvec{0\\1\\1}\\
&= 9\\
\norm{\vec{n_1}} &= \sqrt{4^2+8^2+1^2}\\ 
&= 9\\
\norm{\vec{n_2}} &= \sqrt{0^2+1^2+1^2} \\
&= \sqrt{2}\\
\cos{\theta} &= \frac{9}{9\sqrt{2}}\\
&= \frac{1}{\sqrt{2}}
\end{align}
The planes are inclined at an angle of 45 degrees.
\end{enumerate}
\end{enumerate}
\end{document}